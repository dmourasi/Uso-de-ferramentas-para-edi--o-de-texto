\documentclass[a4paper]{article}


% pacotes

\usepackage[english,brazil]{babel}
\usepackage[letterpaper,top=2cm,bottom=2cm,left=3cm,right=2cm,marginparwidth=2cm]{geometry}
\usepackage{xcolor}
\usepackage{setspace}
\usepackage{enumitem}
\usepackage{ulem}
\usepackage{pifont}


\title{Uso de editores de textos na elabora\c{c}\~{a}o de documentos cient\'{i}ficos}
\author{Daniel Moura Silva}
\date{25 de Setembro de 2023}

\begin{document}
\maketitle

\begin{abstract}
\textbf{{\textcolor{blue}{Instru\c{c}\~{o}es gerais}}} para elabora\c{c}\~{a}o deste documento est\~{a}o apresentadas aqui: foi utilizado 
um documento do tipo artigo com \underline{uma coluna}, papel a4, margem de 2cm (esquerda, direita, superior e inferior); 
comandos para inser\c{c}\~{a}o de \textit{t\'{i}tulo}, \textbf{autor} e \textsc{data} foram utilizados, {\textcolor{red}{fa\c{c}a pesquisa de quais s\~{a}o 
esses comandos e utilize-os!!!}; o \underline{espa\c{c}amento} entre linhas \'{e} duplo. Este texto est\'{a} em uma caixa de 
texto centralizada com largura igual a 75\% da largura do texto. O nome do discente deve ser substitu\'{i}do em “Seu nome”.
\end{abstract}

\section{Introduction}

Your introduction goes here!


\end{document}