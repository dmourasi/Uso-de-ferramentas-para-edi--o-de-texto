\documentclass[a4paper]{article}


% pacotes

\usepackage[english,brazil]{babel}
\usepackage[letterpaper,top=2cm,bottom=2cm,left=2cm,right=2cm,marginparwidth=2cm]{geometry}
\usepackage{xcolor}
\usepackage{setspace}
\usepackage[shortlabels,inline]{enumitem}
\usepackage{ulem}
\usepackage{pifont}


\title{Uso de editores de textos na elabora\c{c}\~{a}o de documentos cient\'{i}ficos}
\author{Daniel Moura Silva}
\date{25 de Setembro de 2023}

\begin{document}
\maketitle

\begin{center}
    \begin{minipage}{0.75\textwidth}
\textbf{{\textcolor{blue}{Instru\c{c}\~{o}es gerais}}} para elabora\c{c}\~{a}o deste documento est\~{a}o apresentadas aqui: foi utilizado 
um documento do tipo artigo com \underline{uma coluna}, papel a4, margem de 2cm (esquerda, direita, superior e inferior); 
comandos para inser\c{c}\~{a}o de \textit{t\'{i}tulo}, \textbf{autor} e \textsc{data} foram utilizados, {\textcolor{red}{fa\c{c}a pesquisa de quais s\~{a}o 
esses comandos e utilize-os!!!}}; o \underline{espa\c{c}amento} entre linhas \'{e} duplo. Este texto est\'{a} em uma caixa de 
texto centralizada com largura igual a 75\% da largura do texto. O nome do discente deve ser substitu\'{i}do em “Seu nome”. {\color{red}\ding{110}}
    \end{minipage}
\end{center}

\begin{flushleft}
Para garantir a format\c{c}\~{a}o os seguintes \colorbox{gray}{pacotes} foram utilizados: \\
    \begin{enumerate*}[(i),itemjoin=\hfill]
        \item {\color{blue}babel} 
        \item {\color{blue}geometry} 
        \item {\color{blue}xcolor}
        \item {\color{blue}setspace} 
        \item {\color{blue}enumitem} 
        \item {\color{blue}ulem} 
        \item {\color{blue}pifont}
    \end{enumerate*}
    \fcolorbox{red}{gray}{\textcolor{green}{!!!Nenhum pacote — al\'{e}m dos descritos acima — deve ser utilizados na elabora\c{c}\~{a}o deste documento!!!}}
\end{flushleft}
\begin{center}
    \begin{minipage}{0.3\textwidth}
        \centering
        \fbox{\parbox{0.8\linewidth}{{Este par\'{a}grafo est\'{a} contido em uma caixa com 4.5 cent\'{i}metros de largura. Voc\^{e} pode acreditar que o texto que est\'{a} acima e abaixo desta
        caixa dista um cent\'{i}metro da caixa}
        }
      }
    \end{minipage}
    \begin{minipage}{0.3\textwidth}
      \centering
      \fbox{\parbox{0.8\linewidth}{{Este par\'{a}grafo est\'{a} contido em uma caixa com 4.5 cent\'{i}metros de largura. Voc\^{e} pode acreditar que o texto que est\'{a} acima e abaixo desta
      caixa dista um cent\'{i}metro da caixa}
      }
    }
    \end{minipage}%
    \begin{minipage}{0.3\textwidth}
      \centering
      \fbox{\parbox{0.8\linewidth}{{Este par\'{a}grafo est\'{a} contido em uma caixa com 4.5 cent\'{i}metros de largura. Voc\^{e} pode acreditar que o texto que est\'{a} acima e abaixo desta
      caixa dista um cent\'{i}metro da caixa}
      }
    }
    \end{minipage}
  \end{center}
  \begin{flushleft}
    A forma como o {\color{blue}\LaTeX} trata o espa\c{c}amento entre as \sout{palavras} \'{e} realmente curioso!\hspace{0.2\textwidth}Este espa\c{c}o
    em branco mede dois cent\'{i}metros
  \end{flushleft}
  \vspace{0.5cm} 

  \noindent
  
  \begin{center}
    \vspace{0.5cm}
    \fcolorbox{red}{green!10}{%
    \parbox{3in}{%
      \color{black} % Cor do texto em preto
      Este par\'{a}grafo est\'{a} contido em uma caixa com tr\^{e}s polegadas de largura. 10\% de verde foi utilizado para preencher o fundo e a cor vermelha
para contornar a caixa. A caixa est\'{a} centralizada e a dist\^{a}ncia entre ela e o texto acima \'{e} de 0.5 cent\'{i}metro.
    }
  }
\end{center}

O fim \hfill do \hfill documento.

{\color{red}\ding{168}} {\color{blue}\hrulefill} {\color{red}\ding{168}}  

\end{document}

